\pagestyle{fancy}
\fancyhead[l]{\autorUO}
\fancyfoot[l]{\asignaturaAbbr}
\fancyfoot[r]{\fecha}

\section{Expectativas y análisis inicial} \label{sec:2}
Antes de comenzar con el análisis de los resultados obtenidos en las diferentes fases del experimento, se establecen 
las expectativas iniciales en cuanto a los tiempos de ejecución y rendimiento de cada una de ellas.

\subsection{Fase 1} \label{subsec:2.1}
En esta primera fase se implementa la multitiplicación de matrices en \python\, así como la creación y manejo de las mimsmas.
Ya que este lenguaje de programación no está optimizado para operaciones de bajo nivel, tal como es el producto 
de matrices, se espera que los tiempos de ejecución sean considerablemente más altos en comparación a los obtenidos 
en las fases posteriores. \\
Será interesante analizar cómo se comportan las diferentes configuraciones de acceso a memoria 
(\rowmajor, \colmajor\ y \zorder) en este contexto, y si existe alguna diferencia significativa respecto a las fases 
2 y 3.

\subsection{Fase 2} \label{subsec:2.2}
En la segunda fase del experimento se implementa la multiplicación de matrices en \C\, mientras que la creación y manejo de las mismas se
realiza en \python. Teniendo esto en cuenta, se espera que los tiempos de ejecución sean mucho más bajos que en la fase 1, 
ya que \C\ es un lenguaje de programación de bajo nivel y está optimizado para este tipo de operaciones. \\
Sin embargo, la creación y manejo de matrices en \python\ puede introducir cierta sobrecarga, lo que podría afectar
los tiempos de ejecución.

\subsection{Fase 3} \label{subsec:2.3}
En esta última fase del experimento, tanto la creación y manejo, como la multiplicación de matrices se realizan en \C.
Se espera que los tiempos de ejecución sean los más bajos de todas las fases, ya que se aprovechan al máximo las
ventajas de \C\ en términos de eficiencia y optimización. \\
En este contexto, sería interesante comparar estos tiempos de ejecución con los obtenidos en la sesión anterior, 
donde la implementación y toma de tiempos se realizó exclusivamente en \C. Debe tenerse en cuenta que, en este caso, 
la toma de tiempos se realiza en \python, lo que podría introducir cierta variabilidad en los resultados.





