\pagestyle{fancy}
\fancyhead[l]{\autorUO}
\fancyfoot[l]{\asignaturaAbbr}
\fancyfoot[r]{\fecha}

\section{Conclusiones} \label{sec:5}
En esta sección se presentan las conclusiones obtenidas durante el desarrollo de la práctica y 
a partir de los resultados obtenidos.


\subsection{Curiosidades}
\subsubsection{\rowmajor\ vs \colmajor} \label{subsec:5.1}
En un principio, se esperaba que el esquema de acceso a memoria \rowmajor\ fuese más eficiente que \colmajor\ debido a la forma en la que se almacenan los datos en memoria.
Sin embargo, los resultados obtenidos muestran lo contrario. Se ha analizado y revisado el código y la implementación de ambos esquemas por 
si existiese algún error, pero no se ha encontrado nada que lo justifique. Incluso se han utilizado herramientas de análisis de rendimiento como \texttt{perf} 
y, de forma contraria a lo analizado de forma empírica en la \autoref{sec:2}, \rowmajor\ produce más fallos de caché que \colmajor.

La única explicación \textit{lógica} a este suceso es la mayor reutilización temporal que se produce en el esquema \colmajor\ respecto a \rowmajor. 
En concreto, esto se debe a que, durante el cálculo de cada columna de la matriz resultado \( C \), se accede repetidamente a una misma columna de la matriz \( B \). 
Dado que la matriz \( B \) está almacenada en memoria en orden por filas, acceder a una columna implica realizar saltos entre elementos que no son contiguos en memoria.
Sin embargo, como esa misma columna se reutiliza varias veces seguidas dentro del mismo cálculo, los datos una vez cargados en caché permanecen disponibles durante varias iteraciones.

A pesar de esta explicación, no se puede asegurar con total certeza que este fenómeno sea la causa principal de la diferencia de rendimiento entre ambos esquemas, 
ya que en la práctica intervienen muchos factores de bajo nivel que pueden influir significativamente en los resultados.


\subsubsection{Tamaños de bloque en \zorder}
Como se comentó en la \autoref{sec:4}, para cada tamaño de matriz se ha probado con todos los tamaños de bloque posibles, incluyendo el tamaño de bloque \(1\) y el del propio tamaño de la matriz.
Esto se ha realizado a propósito para comprobar qué sucede en cada caso. 

La base del algoritmo \zorder\ es la división recursiva de las matrices en submatrices
por lo que el uso de un tamaño de bloque \(1\) no tiene sentido, ya que no se produce dicha división. Esto se traduce en un rendimiento muy deficiente y se ha comprobado con los resultados.
El uso de un tamaño de bloque igual al del tamaño de la matriz tampoco tiene sentido a nivel lógico, pero su rendimiento, si bien no es el mejor, es mucho más óptimo 
que el del caso anterior.  

\subsection{Dificultades}
Durante el desarrollo e implementación de la práctica no se han encontrado dificultades significativas. Sí que es cierto que el hecho de haber realizado 
un programa \textit{extra} para realizar \textit{benchmarks} de forma más simple y rápida ha supuesto un esfuerzo adicional, pero ha simplificado mucho el proceso 
de toma de datos. La mayor dificultad ha sido la de averiguar y analizar por qué \rowmajor\ es más lento que \colmajor, ya que no se esperaba este resultado.
La conclusión respecto a este fenómeno ha sido la comentada en la \autoref{subsec:5.1}.