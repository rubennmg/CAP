\pagestyle{fancy}
\fancyhead[l]{\autorUO}
\fancyfoot[l]{\asignaturaAbbr}
\fancyfoot[r]{\fecha}

\section{Conclusiones} \label{sec:5}
El desarrollo de esta práctica ha permitido analizar en profundidad las diferencias de rendimiento en el cálculo del producto de matrices 
utilizando los lenguajes \python\ y \C, combinando ambos mediante una estrategia de hibridación. Se han cumplido las expectativas iniciales, 
evidenciando que \python\ presenta un rendimiento significativamente inferior al de \C\ en este contexto específico. 

Debe destacarse que esta afirmación no implica que \python\ sea inherentemente ineficiente o un lenguaje ``peor'' que \C. Simplemente, 
su propósito principal no reside en la eficiencia computacional, sino en facilitar la legibilidad, la rapidez de desarrollo y la accesibilidad 
para el programador. Por el contrario, \C\ está diseñado para ofrecer un control más preciso sobre la gestión de memoria y el rendimiento, 
lo que lo hace ideal para tareas computacionalmente intensivas como lo es la multiplicación de matrices.

De hecho, más allá de los análisis realizados, el objetivo principal de esta práctica es hacer notar el valor que aporta la combinación 
de ambos lenguajes en un entorno híbrido. \python\ permite un desarrollo mucho más rápido y sencillo, por lo que es ideal para realizar 
la toma de tiempos (e incluso análisis de los datos obtenidos), mientras que \C, siendo un lenguaje más complejo, es ideal 
para realizar el producto de matrices. \\
Este es el mayor aprendizaje extraído de esta práctica: siempre que sea posible, resulta muy conveniente combinar diferentes lenguajes de 
programación para aprovechar las ventajas de cada uno de ellos, obteniendo así soluciones tanto eficientes como sostenibles en términos 
de desarrollo.

Respecto a dificualtades o problemas encontrados durante la práctica, no se han presentado problemas significativos. La principal 
dificultad ha sido la implementación de los algoritmos de \rowmajor, \colmajor\ y \zorder\ para el producto matricial, pero esto ya 
se había resuleto en la sesión anterior. 