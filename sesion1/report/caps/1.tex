\pagestyle{fancy}
\fancyhead[l]{\autorUO}
\fancyfoot[l]{\asignaturaAbbr}
\fancyfoot[r]{\fecha}

\section{Expectativas y análisis inicial} \label{sec:2}
Antes de analizar los resultados obtenidos, se comentan las expectativas iniciales respecto al rendimiento de cada uno de 
los esquemas de acceso a memoria aplicados al producto de matrices.

\subsection{\rowmajor}
El esquema \rowmajor\ se basa la lectura de matrices por filas, lo que, traspolado al producto de matrices (\( C = A \times B \))
implica que la matriz resultado \textbf{C} se recorrerá de dicha forma. Esto es, para realizar el cálculo de cada elemento de la primera fila de la matriz 
resultado, se recorrerá la primera file de la matriz \textbf{A} y todas las columnas de la matriz \textbf{B}. Este proceso se repetirá para cada una 
de las filas de la matriz \textbf{C}. \\
Se espera que este esquema de acceso a memoria sea muy eficiente, ya que minimiza la cantidad de fallos en caché. Concretamente, 
para el cálculo de cada fila de la matriz \textbf{C}, se espera que se produzcan:
\begin{itemize}
    \item 1 único fallo de caché para la matriz \textbf{C}, al acceder al primer elemento de su primera fila.
    \item 1 único fallo de caché para la matriz \textbf{A}, al acceder al primer elemento de su primera fila.
    \item \(N \times N\) fallos de caché para la matriz \textbf{B}, ya que se accede columna a columna y cada columna tiene \(N\) elementos.
\end{itemize}

\subsection{\colmajor}
El esquema \colmajor\ se basa la lectura de matrices por columnas, lo que, traspolado al producto de matrices (\( C = A \times B \))
implica que la matriz resultado \textbf{C} se recorrerá de dicha forma. Esto es, para realizar el cálculo de cada elemento de la primera columna de la matriz
resultado, se recorrerá la primera columna de la matriz \textbf{B} y todas las filas de la matriz \textbf{A}. Este proceso se repetirá para cada una
de las columnas de la matriz \textbf{C}. \\
Se espera que este esquema de acceso a memoria sea menos eficiente que el anterior, ya que se producen más fallos de caché. Concretamente,
para el cálculo de cada columna de la matriz \textbf{C}, se espera que se produzcan:
\begin{itemize}
    \item \(N\) fallos de caché para la matriz \textbf{C}, ya que se accede a \(N\) elementos de su primera columna.
    \item \(N\) fallos de caché para la matriz \textbf{A}, ya que se accede a \(N\) veces a cada una de sus filas.
    \item \(N \times N\) fallos de caché para la matriz \textbf{B}, ya que se accede \(N\) veces a los \(N\) elementos de su primera columna.
\end{itemize}

\subsection{\zorder}
El esquema \zorder\ se basa en la división recursiva de las matrices en submatrices de tamaño \(2^n \times 2^n\) y su posterior recorrido en orden Z.
Traspolado al producto de matrices (\( C = A \times B \)), implica que la matriz resultado \textbf{C} se recorrerá en orden Z para cada una 
de las submatrices en las que se haya dividido. \\
Este esquema permite una mejor utilización de la memoria caché en casos donde las matrices a multiplicar son de grandes dimensiones, 
ya que se accede a los elementos de forma más local. Sin embargo, su rendimiento puede verse afectado por el tamaño de bloque utilizado.
Se espera, por tanto, que sea el esquema más eficiente en casos de matrices de grandes dimensiones. Predecir empíricamente el número de fallos 
de caché en el uso de este esquema es algo muy complejo, por lo que su estudio se limita a la próxima experimentación y comparación de resultados.
